%%%%%%%%%%%%%%%%%%%%%%%%%%%%%%%%%%%%%%%%%
% Simple Article
% Integrated article template with simple for make4ht
% LaTeX Class
% Version 1.0 (10/11/20)
%
% This class originates by:
% Vel and  Nicolas Diaz
%
% Authors:
% Muhammad Uliah Shafar
%
%
% Free License:
%
%
%%%%%%%%%%%%%%%%%%%%%%%%%%%%%%%%%%%%%%%%%
\documentclass[11pt]{simart} % Font size (can be 10pt, 11pt or 12pt)

%----------------------------------------------------------------------------------------
%	TITLE SECTION
%----------------------------------------------------------------------------------------
% MAIN TITLE SECTION
\title{
\textbf{Lorem Ipsum Lorem Ipsum \\
Lorem Ipsum Lorem Ipsum Lorem Ipsum} \\
\textbf{{Lorem Ipsum Lorem Ipsum \\}}
} % Title and subtitle
%\date{\textbf{\DTMtoday}}
\date{\textbf{\today}}
\author{
\begin{tabular}{@{}ll@{}}
	Nama & : Muhammad Uliah Shafar\\
	NIM & : 21020119420029\\
\end{tabular}
}

%----------------------------------------------------------------------------------------
% OTHER TITLE SECTION

%\title{\textbf{Sistem Sarana dan Prasarana Jl. Pinggir Laut} \\ {\Large\itshape Infrastructure of Waterfront Parepare City}} % Title and subtitle

%\author{\textbf{Uliah Shafar} \\ \textit{Universitas Diponegoro}} % Author and institution

%\date{\today} % Date, use \date{} for no date

%----------------------------------------------------------------------------------------


% \AddToHook{cmd/section/before}{\clearpage} %start each section in new page

\begin{document}
\thispagestyle{empty}
\begin{center}
	\begin{huge}
		\bf{Lorem Ipsum Lorem Ipsum}
	\end{huge}

	\vspace{20pt}
	\includegraphics[width=0.35\textwidth]{logo} \\

	\vspace*{35pt}

	\begin{large}
		\textbf{Lorem Ipsum Lorem Ipsum} \\
		Lorem Ipsum Lorem\\

		\vspace{20pt}
		\textbf{oleh\\
			\vspace{20pt}
			Muhammad Uliah Shafar\\21020119420029}\\

		\vspace{20pt}
		Dosen: \\
		\textbf{Lorem Ipsum}\\



		\vspace{60pt}
		\textbf{PROGRAM STUDI MAGISTER ARSITEKTUR DEPARTEMEN ARSITEKTUR\\
			UNIVERSITAS DIPONEGORO\\
			SEMARANG\\
			2020
		}
	\end{large}
\end{center}
\clearpage

\maketitle % Print the title section

%----------------------------------------------------------------------------------------
%	ABSTRACT AND KEYWORDS
%----------------------------------------------------------------------------------------
\begin{abstract}
	Lorem ipsum dolor sit amet, consectetur adipiscing elit. Curabitur eget faucibus dolor. In posuere, est nec mollis ultrices, ante arcu tristique odio, et rhoncus tortor enim vitae lectus. Aenean auctor enim tempor risus vulputate finibus. Ut quis molestie ex, ut fringilla mauris. Suspendisse ornare sapien nec neque placerat dignissim. Sed vehicula feugiat dolor et blandit. Maecenas convallis diam a lacus faucibus faucibus. Quisque efficitur velit quis lorem consectetur, ac dictum est egestas.

\end{abstract}

\hspace*{3.6mm}\textit{Keywords:} Lorem, Ipsum % Keywords

\vspace{30pt} % Vertical whitespace between the abstract and first section

%----------------------------------------------------------------------------------------
%	ESSAY BODY
%----------------------------------------------------------------------------------------
% adding pdf in altex
% \includepdf[pages={1-2},pagecommand={\thispagestyle{plain}\fakesection{}}]{file.pdf}

\section{Penduluan}%
\label{sec:Penduluan}

Mata kuliah ini memberikan pemahaman dasar tentang gambar teknik sebagai bahasa komunikasi dalam bidang teknik. Mahasiswa dikenalkan pada fungsi, jenis, dan peran gambar teknik serta persiapan dan peralatan menggambar secara manual dan berbasis komputer sesuai standar yang berlaku.

Setelah mempelajari bahan dalam bab ini, seharusnya Anda dapat:
\begin{enumerate}
	\item Menjelaskan fungsi gambar di dalam teknik mesin.
	\item Membedakan berbagai ukuran kertas gambar.
	\item Membuat kertas gambar dengan berbagai ukuran.
	\item Memodifikasi gambar dengan berbagai skala.
	\item Menunjukkan penggunaan berbagai garis gambar.
	\item Mempergunakan etiket standar dalam gambar kerja.
	\item Menyusun etiket gambar untuk gambar lengkap.
	\item Menunjukkan penggunaan huruf dan angka gambar.
	\item Menggambar berbagai konstruksi geometris.
\end{enumerate}

\subsection{Gambar Teknik Sebagai Bahasa Teknik}%
\label{sub:Gambar Teknik Sebagai Bahasa Teknik}

Apabila akan dibuat suatu benda kerja di dalam industri permesinan, maka pemesan atau perencana cukup memberikan gambar kerja pada pelaksana atau teknisi, tidak perlu membawa contoh benda aslinya yang akan dibuat. Hal seperti ini dapat terjadi mengingat gambar dalam teknik dipakai sebagai sarana untuk mengemukakan gagasan tentang konstruksi pekerjaan jadi. Dengan demikian secara ringkas dapat dikatakan bahwa gambar berfungsi sebagai ``bahasa teknik" di industri permesinan.

Untuk dapat melakukan fungsinya sebagai bahasa di industri, maka gambar teknik mesin harus menjadi alat komunikasi utama di antara orang-orang di dalam membuat desain dan komponen industri, bangunan dan peralatan konstruksi, dan pelaksana proyek penghasil permesinan dengan manajemen atau staf ahli permesinan.

Agar dapat melakukan fungsinya sebagai bahasa teknik, maka perlu penguasaan di dalam: (a) penggunaan perkakas gambar, (b) membuat gambar sendiri, dan (c) memahami atau membaca gambar yang dibuat oleh orang lain.
Dari tujuan-tujuan tersebut, maka kemampuan dalam gambar teknik mesin dapat dilihat dari bagaimana ia memahami atau membaca gambar yang dibuat oleh orang lain dan bagaimana kinerjanya dalam membuat gambar agar dapat dipahami oleh orang lain, sedangkan kemampuan penggunaan perkakas gambar sudah termasuk dalam kemampuan membuat gambar, sebab bagaimanapun hasil gambar yang standar pasti diperoleh dari seseorang yang sudah mempunyai keterampilan dalam penggunaan perkakas gambar.
Gambar teknik mesin harus cukup memberikan informasi untuk meneruskan maksud apa yang diinginkan oleh perencana kepada pelaksana, demikian juga pelaksana harus mampu mengimajinasikan apa yang terdapat dalam gambar kerja untuk dibuat menjadi benda kerja yang sebenarnya sesuai dengan keinginan perencana atau pemesan. Untuk itu standar-standar, sebagai tata bahasa teknik, diperlukan untuk menyediakan “ketentuan-ketentuan yang cukup”. Dengan adanya standar-standar yang telah baku ini akan lebih memudahkan suatu pekerjaan untuk dikerjakan di industri pada daerah atau negara lain yang kemudian hasil akhirnya akan dirakit pada industri di daerah atau negara yang berbeda hanya dengan menggunakan gambar kerja.
Agar dapat menggunakan standar-standar gambar yang ada sebagai bahasa, maka gambar teknik yang dibuat harus dapat memberikan pandangan pada bidang yang cukup dan aturan-aturan yang benar, sehingga menunjukkan gambar yang lebih jelas. Selain itu untuk dapat menggunakan gambar sebagai bahasa, orang perlu mempunyai kemampuan: memahami gambar teknik, membuat sketsa-sketsa yang digambar secara bebas atau diagram-diagram detail, penguasaan seluruh lingkup teknik menggambar yang khas bagi gambar kerja dalam lapangan kejuruan yang relevan, dan membuat gambar rancangan \textit{(design)} lengkap.

Meskipun perkembangan teknologi komputer berkembang pesat, sehingga penggambaran yang dilakukan dalam teknik mesin saat sekarang sudah tidak menggunakan pensil, pena gambar \textit{(rapido)}, jangka dan sebagainya, melainkan menggunakan aplikasi program gambar seperti penggunaan AutoCad, Solid Work, Pro Engineering, dan program-program yang lain, namun aturan yang digunakan dalam penggunaan program-program tersebut tetap harus mengacu pada aturan gambar teknik mesin. Jadi dalam penggunaan garis, huruf, proyeksi dan sebagainya tetap berdasarkan aturan gambar teknik mesin.
Sebagai dasar agar nantinya mahasiswa dapat menggunakan gambar sebagai ``bahasa teknik", maka dalam mata kuliah ini tugas-tugas untuk mahasiswa gambarnya dilakukan dengan cara menggunakan pensil dan pena gambar \textit{(rapido)}.

\subsection{Ukuran Kertas Gambar}%
\label{sub:Ukuran Kertas Gambar}

Untuk membuat gambar teknik mesin, dilakukan dengan menggunakan ukuran kertas yang sudah standar. Ada beberapa macam ukuran kertas yang dapat digunakan sesuai dengan kebutuhan dari gambar yang akan dibuat. Ukuran- ukuran kertas tersebut adalah seperti terlihat pada \ref{tab:tab1} berikut ini:

\begin{table}[htb]
	\caption{Ukuran Kertas Gambar}
	\label{tab:tab1}
	\begin{tabularx}{\textwidth}{|c|c|c|c|c|}
		\hline
		Standar  &  Lebar  &  Panjang  &  Tepi kiri  &  Tepi lain  \\
		\hline
		A0       &  841    &  1189     &  20         &  10  \\
		A1       &  594    &  841      &  20         &  10  \\
		A2       &  420    &  594      &  20         &  10  \\
		A3       &  297    &  420      &  20         &  10  \\
		A4       &  210    &  297      &  20         &  5  \\
		A5       &  148    &  210      &  20         &  5  \\
		A6       &  105    &  148      &  20         &  5  \\
		\hline
	\end{tabularx}
\end{table}


Dalam penggunaan kertas gambar untuk membuat gambar kerja tidak bisa dilakukan secara sembarangan, harus dibuat sesuai dengan aturan yang telah ditetapkan, untuk ukuran kertas gambar A3, A2, A1, dan A0, kedudukan kertasnya adalah mendatar (lebar pada arah tegak, dan panjang pada arah datar) seperti terlihat pada \ref{fig:margin1}. Sedangkan untuk ukuran kertas A4, A5, dan A6, kedudukan kertasnya adalah tegak (lebar pada arah datar, dan panjang pada arah tegak) seperti terlihat pada \ref{fig:margin2}.

Ada kalanya karena sesuatu hal pada penggambaran teknik, tidak bisa digambar sesuai dengan ukuran yang sebenarnya, karena misalnya benda yang digambar terlalu kecil, sehingga bila digambar sesuai dengan kenyataan yang sebenarnya tukang yang mengerjakan tidak bisa melihat dengan jelas, dikhawatirkan rusak, atau sebaliknya benda yang digambar terlalu besar, sehingga akan terlalu banyak memakan kertas dan tidak efisien. Maka tukang gambar dapat memperbesar atau memperkecil gambar yang akan dibuat dengan menggunakan skala.

Besar kecilnya skala mempengaruhi efisiensi kerja dan faktor ekonomis. Semakin besar skala akan menyebabkan kertas untuk menggambar menjadi banyak, sehingga diperlukan biaya yang lebih mahal untuk membeli kertas, tinta, dan pengkopiannya, sebaliknya bila skala terlalu kecil dikhawatirkan tidak efisien kerja dan lama dalam penggambaran dan pengerjaan nantinya. Adapun skala untuk pengecilan dan pembesaran yang dinormalisasikan, artinya telah diakui secara internasional untuk gambar teknik mesin adalah sebagai berikut:

\begin{enumerate}
	\item Untuk pengecilan
\end{enumerate}

\begin{tabular}{|c|c|c|}
	\hline
	1 : 2    &  1 : 5    &  1 : 10  \\
	\hline
	1 : 20   &  1 : 50   &  1 : 100  \\
	\hline
	1 : 200  &  1 : 500  &  1 : 1000  \\
	\hline
\end{tabular}

\begin{enumerate}[resume]
	\item Untuk pembesaran
\end{enumerate}

\begin{tabular}{|c|c|c|}
	\hline
	2 : 1  &  5 : 1  &  10 : 1  \\
	\hline
\end{tabular}

\begin{figure}
	\begin{center}
		\includegraphics[width=.95\textwidth]{figures/margin kertas.jpg}
	\end{center}
	\caption{Kedudukan kertas untuk A3 dan di atasnya}
	\label{fig:margin1}
\end{figure}

\begin{figure}
	\begin{center}
		\includegraphics[width=.95\textwidth]{figures/margin kertas a4.jpg}
	\end{center}
	\caption{Kedudukan kertas untuk ukuran A4 dan di bawahnya}
	\label{fig:margin2}
\end{figure}

\begin{tabularx}{\textwidth}{|c|X|c|X|}
	\hline
	\textbf{Bentuk Garis}          &  \textbf{Nama Garis}               &  \textbf{Tebal Garis}  &  \textbf{Penggunaan}  \\
	\hline
	                               &  Garis kontinu (tebal)             &  0,50 -- 0,70          &  Garis benda, garis nyata  \\
	\hline
	                               &  Garis kontinu (tipis)             &  0,25 -- 0,35          &  Garis ukuran, garis bantu, garis ulir, garis arsir, dll.  \\
	\hline
	dash $\approx$ 4 mm, gap 1 mm  &  Garis putus-putus (tebal sedang)  &  0,35 -- 0,50          &  Garis bayang-bayang  \\
	\hline
	dash $\approx$ 7 mm, gap 1 mm  &  Garis titik garis (tebal)         &
	\begin{tabular}[c]{@{}c@{}}0,50 -- 0,70\\0,25 -- 0,35\end{tabular}
	                               &  Garis potong  \\
	\hline
	dash $\approx$ 7 mm, gap 1 mm  &  Garis titik garis (tipis)         &  0,25 -- 0,35          &  Garis sumbu, garis lipatan  \\
	\hline
	                               &  Garis bebas (tipis)               &  0,25 -- 0,35          &  Garis potong  \\
	\hline

	dash $\approx$ 7 mm, gap 1 mm  &  Garis titik dua garis (tipis)     &  0,25 -- 0,35          &  Garis bagian bergerak, garis di depan bidang potong, garis bentuk awal, dll.  \\
	\hline
\end{tabularx}

Ketebalan garis gambar di atas sudah standar, tetapi bisa juga di dalam pemakaiannya tukang gambar hanya menggunakan perkiraan di dalam menetapkan garis gambar yang digunakan, keadaan seperti ini dapat timbul jika gambar-gambar  yang  dibuat  terlalu  kecil  atau  komponen-komponen  yang digambar terlalu banyak, sehingga apabila dibuat garis sesuai aturan, mungkin timbul kesan gambarnya menjadi kurang sesuai atau mungkin menjadi sempit. Untuk menghindari kesan-kesan tersebut maka tebal garis, dibuat dengan menggunakan perbandingan seperti di bawah ini.

\begin{tabularx}{0.6\textwidth}{|c|X|}
	\hline
	  &  Garis tebal ($s$)  \\
	\hline
	  &  Garis tipis ($\frac{1}{4}s$)  \\
	\hline
	  &  Garis tipis bergelombang ($\frac{1}{4}s$)  \\
	\hline
	  &  Garis putus-putus ($\frac{1}{2}s$)  \\
	\hline
	  &  Garis putus-putus campur tipis ($\frac{1}{4}s$)  \\
	\hline
	  &  Garis strip titik dengan ujung tebal ($s$ dan $\frac{1}{4}s$)  \\
	\hline
	  &  Garis putus-putus campur tebal ($s$)  \\
	\hline
\end{tabularx}

Untuk memperjelas penggunaan dari masing-masing jenis garis tersebut, dapat dilihat \ref{fig:jenisgaris}. Pada gambar tersebut nampak bahwa masing-masing jenis garis digunakan sesuai dengan fungsinya seperti yang telah dijelaskan.

\begin{figure}
	\begin{center}
		\includegraphics[width=.95\textwidth]{figures/jenis garis.jpg}
	\end{center}
	\caption{Penggunaan macam-macam jenis garis}
	\label{fig:jenisgaris}
\end{figure}

\subsection{Etiket Gambar}%
\label{sub:Etiket Gambar}









---
\section{Pendahuluan}

This file powered by \citep{einstein} and \citep{latexcompanion}. It can also be found in \cite{knuthwebsite}.

\begin{figure}[htpb]
	\centering
	\includegraphics[width=0.8\textwidth]{placeholder}
	\caption{placeholder.jpg}
	\caption*{Sumber: Dokumen Pribadi, 2020}
	\label{fig:placeholder-jpg}
\end{figure}
\lipsum[1-3]

\subsection{Latar Belakang}

\lipsum[4-5]


\subsubsection{Definisi}

\lipsum[6-7]
\begin{table}[h] % [h] forces the table to be output where it is defined in the code (it suppresses floating)
	\caption{Example table.}
	\centering
	\begin{tabular}{l l r}
		\toprule
		\multicolumn{2}{c}{Name}  \\
		\cmidrule(r){1-2}
		First Name  &  Last Name  &  Grade  \\
		\midrule
		John        &  Doe        &  $7.5$  \\
		Richard     &  Miles      &  $5$  \\
		\bottomrule
	\end{tabular}
\end{table}



\subfile{subfiles/subfile.tex}

\section{Metodologi Penelitian}

\lipsum[15-19]

%\begin{figure}[htbp]
%\centering
%\begin{subfigure}{6cm}
%\centering\includegraphics[width=5cm]{placeholder.jpg}
%\caption{}
%\end{subfigure}%
%\begin{subfigure}{6cm}
%\centering\includegraphics[width=5cm]{placeholder.jpg}
%\caption{}
%\end{subfigure}\vspace{10pt}
%
%\caption{Lorem Ipsum}
%\label{}
%\end{figure}

%----------------------------------------------------------------------------------------
%	BIBLIOGRAPHY
%----------------------------------------------------------------------------------------
% Comment these out if they are already in sub
\bibliographystyle{apalike}

\bibliography{biblio.bib}



\end{document}
